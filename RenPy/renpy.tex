% Copyright (c) 2010 Milton Mazzarri <milmazz@gmail.com>
% All rights reserved.
% 
% Redistribution and use in source and binary forms, with or without
% modification, are permitted provided that the following conditions
% are met:
% 1. Redistributions of source code must retain the above copyright
%    notice, this list of conditions and the following disclaimer.
% 2. Redistributions in binary form must reproduce the above copyright
%    notice, this list of conditions and the following disclaimer in the
%    documentation and/or other materials provided with the distribution.
% 3. The name of the author may not be used to endorse or promote products
%    derived from this software without specific prior written permission.
% 
% THIS SOFTWARE IS PROVIDED BY THE AUTHOR ``AS IS'' AND ANY EXPRESS OR
% IMPLIED WARRANTIES, INCLUDING, BUT NOT LIMITED TO, THE IMPLIED WARRANTIES
% OF MERCHANTABILITY AND FITNESS FOR A PARTICULAR PURPOSE ARE DISCLAIMED.
% IN NO EVENT SHALL THE AUTHOR BE LIABLE FOR ANY DIRECT, INDIRECT,
% INCIDENTAL, SPECIAL, EXEMPLARY, OR CONSEQUENTIAL DAMAGES (INCLUDING, BUT
% NOT LIMITED TO, PROCUREMENT OF SUBSTITUTE GOODS OR SERVICES; LOSS OF USE,
% DATA, OR PROFITS; OR BUSINESS INTERRUPTION) HOWEVER CAUSED AND ON ANY
% THEORY OF LIABILITY, WHETHER IN CONTRACT, STRICT LIABILITY, OR TORT
% (INCLUDING NEGLIGENCE OR OTHERWISE) ARISING IN ANY WAY OUT OF THE USE OF
% THIS SOFTWARE, EVEN IF ADVISED OF THE POSSIBILITY OF SUCH DAMAGE.

\documentclass{beamer}

\mode<presentation>
{
  \usetheme{Darmstadt}
  \setbeamercovered{transparent}
  \useoutertheme{tree}
  \useinnertheme{rounded}
}

\usepackage[spanish]{babel}
\usepackage[utf8]{inputenc}

\usepackage{times}
\usepackage[T1]{fontenc}

% Redefinicion de teorema, definicion, ejemplo
\newtheorem{teorema}[theorem]{Teorema}
\newtheorem{definicion}[theorem]{Definición}
\newenvironment{ejemplo}{\begin{exampleblock}{Ejemplo}}{\end{exampleblock}}

 \AtBeginSection[]
{
  \begin{frame}<beamer>[allowframebreaks]{Contenido}
    \tableofcontents[currentsection,currentsubsection]
  \end{frame}
}

\title{Maestro Virtual}
\subtitle{Novela gráfica creada con el motor Ren'Py}
\author[Milton Mazzarri]{Milton Mazzarri \\
\href{mailto:milmazz@gmail.com}{milmazz@gmail.com} \\
}
\institute[gUsLA]{Grupo de Usuarios de Software Libre de la Universidad de Los Andes}
%
\date{11 de Julio de 2008}

\hypersetup{colorlinks=true,linkcolor=blue}

\hypersetup{
pdfauthor={Milton R. Mazzarri S.},
pdftitle={Maestro Virtual: Novela gráfica creada con el motor Ren'Py}
}

\beamerdefaultoverlayspecification{<+->}

\begin{document}

\begin{frame}
\titlepage
\end{frame}

\section{Maestro Virtual}

\subsection{Origen}

\frame{
  \frametitle{El origen del proyecto}

  \begin{itemize}
    \item
      El \alert{Maestro Virtual} nace como colaboración con el proyecto
      educativo del grupo de investigación en Sistemología Interpretativa de la
      Universidad de Los Andes.
    \item
      El proyecto educativo está basado en la enseñanza a los niños a través de
      narraciones.
    \item
      La estrategia es incluir los contenidos en el mismo cuento, aprovechando
      la narración de sucesos para incluir actividades de enseñanza. 
    \item
      El propósito es hacer uso del conocimiento para enriquecer la historia.
  \end{itemize}
}

\subsection{Contenidos}

\frame{
  \frametitle{El contenido de los cuentos}

  \begin{itemize}
    \item
      La Dra. Miriam Villarreal (ULA) diseñó los contenidos que dicen paso a
      paso como realizar las actividades en torno al cuento. 
    \item
      En este caso, \emph{El Príncipe Feliz} de Oscar Wilde.
    \item
      El diseño del cuento de \emph{El Príncipe Feliz} permitió crear una guía
      para niños de quinto grado de primaria.
  \end{itemize}
}

\subsection{Uso de Tecnología de Información}

\frame{
  \frametitle{Inclusión de tecnologías de información}

  \begin{itemize}
    \item
      El proyecto \alert{Maestro Virtual} pretende desarrollar una aplicación
      de \emph{software} cuyo guión sea el contenido para quinto grado diseñado
      sobre el cuento \emph{El Príncipe feliz}.
    \item
      Para lograr esto se comenzó a observar y estudiar los principios
      educativos en torno al proyecto educativo, seguido de una investigación
      acerca de las herramientas \alert{libres} existentes para poder
      desarrollar la aplicación de manera que exista una interacción entre el
      niño y la máquina.
    \item
      El \emph{software} \alert{Maestro Virtual} no pretende sustituir al
      maestro, será un apoyo. El hecho de usar el computador como herramienta
      atraerá la atención del niño.
  \end{itemize}
}

\section{Ren'Py}

\subsection{Motor de Ren'Py}

\frame{
  \frametitle{Ren'Py: ¿Qué es exactamente?}

  \begin{itemize}
    \item
      Es un motor que soporta la creación de novelas gráficas, simuladores de
      citas y distintas formas de contar historias a través de un ordenador.
    \item
      El motor permite crear novelas personalizables y flexibles.
    \item
      Ofrece opciones de cualquier novela gráfica.
      \begin{itemize}
        \item
          Guardar, cargar partidas.
        \item
          Preferencias.
        \item
          Retroceder escenas.
        \end{itemize}
    \item
      La licencia de Ren'Py permite crear novelas comerciales y no-comerciales.
    \item
      Ren'Py funciona en Linux, Mac y \emph{Winbugs}.
  \end{itemize}
}

\subsection{Funcionalidades}
\begin{frame}[fragile]
  \frametitle{Diálogos, Narraciones y Lógica de la novela}

  \begin{itemize}
    \item<1->
      La sintaxis de las novelas gráficas usando Ren'Py son muy similares al
      formato de un guion de una película.
    \item<2->
      Ren'Py necesita de poco código para que tus personajes realmente digan
      algo.
  \end{itemize}

  \begin{ejemplo}<3->
\begin{semiverbatim}
\tiny
# script.rpy
init:
    $ pebbels = Character("Pebbels", color="#c8ffc8")

label start
    "Hace un año..."
    pebbels "Gazoo"
    "Bamm-Bamm" "Bam Bam..."
\end{semiverbatim}
  \end{ejemplo}
\end{frame}

\frame{
  \frametitle{Diálogos, Narraciones y Lógica de la novela}
 
  \begin{itemize}
    \item
      En los scripts de Ren'Py los personajes están referenciados por nombre,
      no por número.
    \item
      Ren'Py ajusta el texto de las narraciones o diálogos para que encaje en
      la proyección de pantalla.
    \item
      La lógica de la novela puede ser basada en menú de opciones o variables,
      de ese modo una novela gráfica puede tener N finales, mejorando la
      experiencia del usuario.
    \item
      La presentación de las narraciones es \alert{totalmente} personalizable,
      ubicación, colores, estilos, fuentes tipográficas, imágenes.
    \item
      El texto puede incluir etiquetas.
  \end{itemize} 
}

\frame{
  \frametitle{Interfaz de usuario y estado de la novela gráfica}

  \begin{itemize}
    \item
      Soporta tanto guardar/cargar sesiones automáticamente.
    \item
      Proporciona múltiples \alert{slots} para almacenar estados.
    \item
      Interacción con el mouse, teclado y \alert{joystick}.
    \item
      Puede programar el avance automático de las escenas, de ese modo los
      usuarios no necesitan dar clic para avanzar.
  \end{itemize}
}

\frame{
  \frametitle{Efectos especiales}

  \begin{itemize}
    \item
      Soporte de modo pantalla completa o en una ventana.
    \item
      Un número ilimitado de imágenes puede ser mostradas en una escena.
    \item
      Se pueden reproducir vídeos \texttt{mpeg1}.
    \item
      Soporta la precarga de imágenes, de ese modo se evitan los retardos en la
      presentación de las escenas.
    \item
      Proporciona efectos en las transiciones de las escenas.
  \end{itemize}
}

\frame{
  \frametitle{Sonidos}

  \begin{itemize}
    \item
      Reproducción de ficheros \texttt{mp3}, ogg vorbis, entre otros.
    \item
      Efectos aparecer/desvanecer progresiva de audio.
    \item
      Soporte para voces (actuación de los diálogos y narraciones).
    \item
      Efectos de sonido, música y voces puede reproducirse simultáneamente.
  \end{itemize}
}

\frame{
  \frametitle{Distribución}

  \begin{itemize}
    \item
      Soporta Linux, Mac y el \emph{innombrable}. Soporte para otras
      plataformas es posible.
    \item
      Ren'Py es distribuido bajo la licencia MIT, eso quiere decir que puede
      vender las novelas hechas sin necesidad de pagar por Ren'Py. La
      distribución libre obviamente es permitida.
    \item
      Los scripts de Ren'Py puede ser ofuscados para su distribución, haciendo
      difícil la lectura o modificación del código fuente.
    \item
      Los ficheros de medios pueden ser archivados para la distribución de tal
      modo que sea difícil acceder a los ficheros individualmente.
  \end{itemize}  
}

\subsection{Comenzando el desarrollo}
\frame{
  \frametitle{Desarrollo de novelas}

  \begin{columns}

  \begin{column}{5cm}
    \begin{itemize}
      \item<1->
        La distribución de Ren'Py viene con un lanzador que le ayuda a comenzar
        y manejar sus proyectos.
      \item<3->
        Ren'Py le ofrece algunas plantillas para facilitarle el comienzo.
      \item<5->
        La documentación de Ren'Py se encuentra en el sitio oficial.
    \end{itemize}
  \end{column}

  \begin{column}{5cm}
    \begin{overprint}
      \includegraphics<2>[scale=0.5]{pix/launcher}
      \includegraphics<4>[scale=0.5]{pix/template}
      \includegraphics<6>[scale=0.3]{pix/renpy-site}
    \end{overprint}
  \end{column}

  \end{columns}
}

\subsection{¿Cómo encontrar ayuda?}
\frame{
  \frametitle{Ayuda en línea}

  \begin{itemize}[<+->]
      \item
        Los foros de
        \href{http://lemmasoft.renai.us/forums/viewforum.php?f=8}{Lemmasoft} le
        pueden ayudar en el desarrollo de las novelas.
      \item
        Ayuda en línea en el canal IRC \alert{\#renpy} en
        \href{http://freenode.net/}{freenode.net}
      \item
        Revisión del código fuente disponible para las novelas gráficas. Hasta
        ahora son más de \alert{110} novelas hechas con Ren'Py.
  \end{itemize}
}

\section{Involucrados en el Proyecto Maestro Virtual}
\frame{
  \frametitle{Autores}

  Hasta ahora los involucrados en el Proyecto \alert{Maestro Virtual} son:

  \begin{itemize}[<1->]
    \item
      José Aguilar --
      \href{mailto:jaguilar@cenditel.gob.ve}{jaguilar@cenditel.gob.ve} --
      Asesor 
    \item
      Ramsés Fuenmayor -- \href{ramsesfa@ula.ve}{ramsesfa@ula.ve} -- Asesor
    \item
      Ana Rangel --
      \href{mailto:arangel@cenditel.gob.ve}{arangel@cenditel.gob.ve} --
      Investigación, voces.
    \item
      Luz Mairet Chourio
      \href{mailto:lchourio@cenditel.gob.ve}{lchourio@cenditel.gob.ve} --
      Investigación.
    \item
      Iván Dario Hernández --
      \href{mailto:ivandarioh@yahoo.es}{ivandarioh@yahoo.es} -- Arte.
    \item
      Milton Mazzarri --
      \href{mailto:milmazz@gmail.com}{milmazz@gmail.com} --
      Programador, voces.
  \end{itemize}  
}

\section{Herramientas utilizadas en el desarrollo}
\frame{
  \frametitle{Herramientas utilizadas}
  
  \begin{columns}
  
  \begin{column}{5cm}
    \begin{itemize}
      \item<1->
        Ren'Py
      \item<2->
        Gimp
      \item<3->
        Inkscape
      \item<4->
        Audacity
      \item<5->
        Vim
    \end{itemize}
  \end{column}

  \begin{column}{5cm}
    \begin{overprint}
      \includegraphics<1>{pix/renpy}
      \includegraphics<2>{pix/wilber}
      \includegraphics<3>{pix/inkscape}
      \includegraphics<4>{pix/audacity}
      \includegraphics<5>{pix/vim}
    \end{overprint}
  \end{column}
  \end{columns}
}

\section{Enlaces de interés}

\frame{
    \frametitle{Enlaces de interés}

    \begin{itemize}
        \item Seguimiento del Proyecto:
        \url{http://maestro.cenditel.gob.ve/trac/}
        \item Código fuente:
        \url{http://maestro.cenditel.gob.ve/svn/renpy-maestrovirtual}
        \item Lista de discusión:
        \url{http://listas.cenditel.gob.ve/listinfo/maestrovirtual}
    \end{itemize}
}

\frame{
    \frametitle{¡Muchas Gracias!}

    \begin{description}[<1->]
        \item[Nombre] Milton Mazzarri
        \item[Correo] \href{mailto:milmazz@gmail.com}{milmazz@gmail.com}
        \item[Blog] \url{http://blog.milmazz.com.ve}
        \item[Twitter] \url{http://twitter.com/milmazz}
        \item[Presentaciones] \url{http://www.scribd.com/milmazz}
    \end{description}
}

\end{document}
