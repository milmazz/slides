\documentclass{beamer}  
\mode<presentation>
{
\usetheme{JuanLesPins}
}
\beamertemplatenavigationsymbolsempty

\usepackage[spanish]{babel}
\usepackage[utf8]{inputenc}

\newtheorem{definicion}[theorem]{Definición}
\newenvironment{ejemplo}{\begin{exampleblock}{Ejemplo}}{\end{exampleblock}}

\hypersetup{colorlinks=true,linkcolor=blue,urlcolor=blue}

\title{Conociendo el \emph{framework Web} Django}
\subtitle{\emph{Framework Web} para los perfeccionistas con plazos de entrega}
\author[Milton Mazzarri]{Milton Mazzarri \\
\href{mailto:milmazz@gmail.com}{milmazz@gmail.com} \\
}
\institute[gUsLA, VELUG, Debian-ve]{Grupo de Usuarios de Software Libre de la Universidad de Los
Andes (gUsLA)\\
Debian Venezuela\\
VELUG
}
\date{Febrero 26, 2007}

\begin{document}

\begin{frame}[plain]
  \titlepage
\end{frame}

\section{Introducción}
\frame{
\frametitle{Conozca Django}
\alert{Django} es un \emph{framework} de alto nivel para el desarrollo
Web escrito en \alert{Python} que promueve un desarrollo ágil mientras
se mantiene un diseño limpio y pragmático.
}

\section{Python}
\subsection{Características}
\begin{frame}
  \frametitle{¿Qué es Python?}
  \begin{itemize}[<+->]
    \item
      Lenguaje de programación orientado a objetos, interactivo e
      interpretado.
    \item
      Estructuras de datos de alto nivel, dinámicamente tipado.
    \item
      Aumenta la productividad de desarrollo en comparación con lenguajes
      estáticamente tipados.
    \item
      Multiplataforma:
      \begin{itemize}
        \item
          El interprete de \alert{CPython} puede ejecutarse nativamente en
	  muchas plataformas (esto incluye \alert{GNU/Linux}).
        \item
          \alert{Jython} corre dentro de \alert{JVM}
        \item
          \alert{IronPython} está orientado a plataformas \alert{.Net} y
	  \alert{Mono}.
      \end{itemize}
  \end{itemize}
\end{frame}

\subsection{Ventajas}
\begin{frame}
  \frametitle{¿Por qué usar Python?}
  \begin{itemize}[<+->]
    \item
      Calidad del Software.
    \item
      Incrementa la productividad en el desarrollo.
    \item
      Portabilidad.
    \item
      Integración de componentes.
    \item
      Diversión.
  \end{itemize}
\end{frame}

%\begin{frame}[fragile]
%  \frametitle{Zen de Python}
%  \begin{definicion}
%    \begin{verbatim}
%      >>> import this
%      The Zen of Python, por Tim Peters
%
%      Hermoso es mejor que Feo. 
%      Explícito es mejor que Implícito.
%      Simple es mejor que Complejo.
%      Complejo es mejor que Complicado.
%      Lineal es mejor que Cruzado.
%      Escaso es mejor que Denso. 
%      Poder leer el código, cuenta.
%      Los Casos Especiales no son tan especiales como para romper reglas.
%      Lo Práctico le gana a Lo Puro.
%    \end{verbatim}
%  \end{definicion}
%\end{frame}

%\begin{frame}[fragile]
%  \frametitle{Zen de Python}
%  \begin{definicion}
%    \begin{verbatim}
%      Los Errores no deben pasar desapercibidos. 
%      A menos de que Explícitamente se silencien.
%      En la ambigüedad, rehusa el Adivinar. 
%      Sólo debería haber una, y preferiblemente Una, manera obvia de hacer las cosas.
%      Aunque esa manera no se obvia al principio, a menos que seas (Dutch).
%      Ahora es mejor que Nunca.
%      A pesar de que Nunca es mejor que Ahora Mismo.
%      Si la implementación es difícil de explicar, es probable que sea una mala idea. 
%      Si la implementación es fácil de explicar, es probable que sea una buena idea.
%      Namespaces son una Gran idea — Hagamos mas de esas.
%    \end{verbatim}
%  \end{definicion}
%\end{frame}

\subsection{¿Para qué sirve?}
\begin{frame}
  \frametitle{Tipo de aplicaciones}
  \begin{itemize}[<+->]
    \item
      Sistemas Operativos
    \item
      Interfaces gráficas de usuario
    \item
      Internet
    \item
      Bases de datos
    \item
      Juegos
    \item
      Procesamiento de imágenes
    \item
      Inteligencia Artificial
    \item
      XML
    \item
      \ldots
  \end{itemize}
\end{frame}

\subsection{Tiene que haber algo malo, ¿no?}
\begin{frame}
  \frametitle{El problema de la velocidad}
  \begin{itemize}[<+->]
    \item
      Quizá la única desventaja de \alert{Python} es esa, como está implementado
      actualmente, la velocidad de ejecución puede no ser tan rápida como la
      de los lenguajes compilados.
    \item
      Puede ser cierto que \alert{Python} presente problemas de velocidad de ejecución
      respecto a los lenguajes compilados, pero su principal objetivo es la
      productividad de desarrollo, no la velocidad de procesamiento.
    \item
      En aquellos casos donde el rendimiento es crítico, usted puede crear
      módulos en \alert{C} y luego integrarlos.
    \item
      El proyecto \alert{psyco} (\emph{JIT Compiler}) permite mejorar sustancialmente el
      rendimiento del código \alert{Python}.
  \end{itemize}
\end{frame}

\subsection{¿Quienes han tenido el placer de conocerlo?}
\begin{frame}
  \frametitle{¿Quienes usan Python?}
  \begin{itemize}[<+->]
    \item
      AstraZeneca
    \item
      Honeywell
    \item
      Industrial Light \& Magic
    \item
      \url{http://www.python.org/about/success/}
    \item
      \url{http://www.python.org/about/quotes/}
  \end{itemize}
\end{frame}

\subsection{Aplicación de Python en nuestra área de interés}
\begin{frame}
  \frametitle{Desarrollo \emph{Web} en Python}
  \begin{itemize}[<+->]
    \item
      Después de varios años, muchas opciones, y sin ningún framework líder, finalmente algunos proyectos comienzan a ganar adeptos.
      \begin{itemize}
        \item
          Zope (Plone)
        \item
          TurboGears
	\item
	  Pylons
        \item
          \alert{Django}
      \end{itemize}
  \end{itemize}
\end{frame}

\section{Django}
\subsection{¿Por qué Django fue desarrollado?}
\begin{frame}
  \frametitle{Historia de Django}
  \begin{itemize}[<+->]
    \item
      Originalmente desarrollado para manejar 3 sitios,
      \href{http://ljworld.com}{LJWorld.com} (noticias),
      \href{http://lawrence.com}{Lawrence.com} (entretenimiento y
      música) y \href{http://kusports.com}{KUSports.com} (deportes).
    \item
      Adrian Holovaty, Simon Willison, Jacob Kaplan-Moss y Wilson Miner
      desarrollaron \alert{Django} cuando trabajaban para \alert{World
      Company} en Lawrence, Kansas. 
    \item
      El proyecto comienza \alert{dentro} de la empresa a finales del año
      2003. Desde el principio el proyecto busca darle solución a los
      problemas que a diario se les presentaban al equipo principal de
      desarrollo Web.
    \item
      En Julio de 2.005 se libera al público bajo la licencia BSD.
    \item
      La versión estable actual es la \alert{0.95.1}. La compatibilidad
      con versiones anteriores no está garantizada hasta la llegada de
      la versión \alert{1.0}.
  \end{itemize}
\end{frame}

\subsection{Filosofía de diseño}
\begin{frame}
\frametitle{Filosofía de diseño}
\begin{itemize}
\item<1-> \alert{Acoplamiento}: Grado de mutua interdependencia entre de
módulos/componentes. Bajo acoplamiento es mejor.
\item<2-> \alert{Cohesión} Grado de responsabilidades que forman una unidad
significativa en un módulo/componente. Mayor cohesión es mejor.
\item<3-> \alert{Menos código} Las aplicaciones en Django usan la menor cantidad de
código posible. Django aprovecha las ventajas que ofrece el lenguaje Python.
\item<4-> \alert{Desarrollo ágil} La idea de los \alert{frameworks} del siglo XXI es
hacer los aspectos tediosos del desarrollo Web rápidos.
\item<5->\alert{DRY} \emph{Don't Repeat Yourself}
\item<6-> \alert{Explícito es mejor que implícito} Principio fundamental del núcleo del
lenguaje \alert{Python}.
\item<7-> Consistencia.
\end{itemize}
\end{frame}


\subsection{Algunos casos específicos de uso}
\frame{
  \frametitle{¿Quienes usan Django?}
  \begin{itemize}[<+->]
    \item
      World Online usa \alert{Django} en todos sus sitios, tanto
      para sitios internos como clientes.
      \begin{itemize}
        \item 
	  \url{http://www.lawrence.com/}
	\item
          \url{http://www.6newslawrence.com/}
        \item
	  \url{http://www.visitlawrence.com/}
        \item
	  \url{http://www.lawrencechamber.com/}
        \item
	  \url{http://www2.kusports.com/stats/}
      \end{itemize}
    \item
      \href{http://washingtonpost.com}{Washington Post} usa Django en
      algunos de sus proyectos, algunos ejemplos son los siguientes:
      \begin{itemize}
        \item
	  \url{http://projects.washingtonpost.com/congress/}
	\item
	  \url{http://projects.washingtonpost.com/fallen/}
      \end{itemize}
  \end{itemize}
}

%\subsection{Algunos casos específicos de uso}
\frame{
  \frametitle{¿Quienes usan Django?}
  \begin{itemize}[<+->]
    \item
      \url{http://chicagocrime.org}, permite acceder libremente a la
      base de datos de crimenes reportados en Chicago (EEUU), usa de
      manera original el API de \alert{Google Maps}.
    \item
      \url{http://tabblo.com}, un innovador sitio para compartir
      fotografías, al estilo de \href{http://flickr.com/}{Flickr}.
    \item
      \url{grono.net}, una red social de más de 450.000 usuarios en
      Polonia, se reemplazó el código en \alert{Java} por \alert{Django}. Los
      desarrolladores encontraron no solo un mayor rendimiento en la
      aplicación , sino que era divertido trabajar en Django, además,
      requería de menos \emph{hardware}.
  \end{itemize}
}

%\subsection{Algunos casos específicos de uso}
\frame{
  \frametitle{¿Quienes usan Django?}
  \begin{itemize}[<+->]
    \item
      \url{traincheck.com}, el sitio le permite enviar mensajes de texto
      desde su móvil celular con el fin de obtener información acerca de
      los horarios del tren subterráneo en su localidad.
    \item
      Una lista completa y actualizada de los sitios que utilizan
      \alert{Django} puede encontrarla en
      \url{http://code.djangoproject.com/wiki/DjangoPoweredSites}
  \end{itemize}
}

\subsection{El origen del nombre del proyecto}
\begin{frame}
  \frametitle{¿Qué significa Django?}
  \begin{itemize}[<+->]
    \item
      El nombre del framework se da en honor a \alert{Django Reinhardt}, gitano cuya profesión 
      fue ser guitarrista de Jazz a partir de los años 30 hasta el comienzo de la década 
      de los 50. Actualmente es considerado uno de los mejores guitarristas de todos los tiempos.
    \item
      De acuerdo a la sección de
      \href{http://www.djangoproject.com/documentation/faq/}{preguntas
      frecuentes} del proyecto, \alert{Django} se pronuncia
      \alert{JANG-oh}, la letra \alert{D} la obviáremos.
  \end{itemize}
\end{frame}

%\subsection{El origen del nombre del proyecto}
\begin{frame}
  \frametitle{¿Qué significa Django?}
  \begin{itemize}[<+->]
    \item
      Adrian Holovaty, uno de los creadores del framework, le gusta tocar guitarra, quizá esto 
      pudo haber influido directamente en el nombre del proyecto.
    \item
      Si llega a convertirse en un aficionado de Django, seguramente le gustaría conocer acerca 
      de la lista de otros nombres que consideraron para el proyecto,
      lea el artículo escrito por Jacob Kaplan-Moss, 
      \href{http://www.jacobian.org/writing/2005/sep/09/private_dancer/}{Private
      Dancer?}
  \end{itemize}
\end{frame}

\subsection{Primeros pasos}
\begin{frame}
  \frametitle{¿Cómo comenzar?}
  \begin{itemize}[<+->]
    \item
      Sitio principal: \url{http://www.djangoproject.com}
    \item
      Grupos en Google:
      \begin{description}
        \item
	  [users] Grupo de desarrolladores que utilizan
	  \alert{Django} para construir sitios web.
	  \url{http://groups.google.com/group/django-users}
        \item
	  [developers] Grupo de usuarios que activamente trabajan con
	  el código de \alert{Django}.
	  \url{http://groups.google.com/group/django-developers}
      \end{description}
    \item
      Usted puede obtener información acerca del repositorio \alert{SVN}
      y cómo enviar correcciones al código en el sitio:
      \url{http://code.djangoproject.com/}
  \end{itemize}
\end{frame}

%\begin{frame}
%\frametile{Recursos en español}
%\begin{itemize}[<+->]
%\item
%Listas de correo:
%\begin{itemize}
%\item
%          http://listas.aditel.org/listinfo/python-es
%\item
%          http://python.org.ve/mailman/listinfo/pyve
%\end{itemize}
%\item
%    IRC:
%\begin{itemize}
%\item
%          #python-es en Freenode
%\item
%          #python-ve en Freenode
%\end{itemize}
%\end{itemize}
%\end{frame}

\subsection{Instalación de Django}
\begin{frame}[fragile]
  \frametitle{Versión oficial}
  \begin{itemize}[<+->]
    \item
      La versión oficial más reciente la obtenemos desde
      \url{http://www.djangoproject.com/download/}
    \item
      \alert{Django} usa el método de instalación estándar en
      \alert{Python}: \texttt{distutils}
  \end{itemize}
  \begin{ejemplo}<3->
    \begin{verbatim}
      $ tar xvzf Django-*.tar.gz
      $ cd Django-*
      $ sudo python setup.py install
    \end{verbatim}
  \end{ejemplo}
\end{frame}

\begin{frame}[fragile]
  \frametitle{Django desde subversion}
  \begin{ejemplo}
    \begin{verbatim}
      $ sudo aptitude install subversion
      $ svn co\
      http://code.djangoproject.com/svn/django/trunk\
      django_src
      $ sudo ln -s `pwd`/django_src/django\
      /usr/lib/python2.4/site-packages/django
      $ sudo update-alternatives \
      /usr/bin/django-admin.py \
      django-admin.py \
      /usr/lib/.../.../django/bin/django-admin.py 10
    \end{verbatim}
  \end{ejemplo}
\end{frame}

\subsection{Estableciendo una base de datos}
\frame{
  \frametitle{Bases de datos}
  \begin{itemize}[<+->]
    \item
      El único pre-requisito de Django es tener instalado Python. Sin
      embargo, es muy probable que necesitemos almacenar nuestros datos en
      una base de datos.
    \item
      Django soporta cinco (5) Sistemas de Gestión de Bases de Datos.
      \begin{itemize}
        \item
	  PostgreSQL
	\item
	  SQLite 3
	\item
	  MySQL
	\item
	  Microsoft SQL Server
	\item
	  Oracle
      \end{itemize}
  \end{itemize}
}

\subsection{Usando Django con PostgreSQL}
\begin{frame}[fragile]
  \frametitle{Trabajando con PostgreSQL}
  \begin{ejemplo}
    \begin{verbatim}
      $ sudo aptitude install postgresql\
      python-psycopg 
    \end{verbatim}
  \end{ejemplo}
\end{frame}

\section{Demostración}
\begin{frame}
  \frametitle{Ahora comienza la diversión}
  \begin{itemize}[<+->]
    \item
      Iniciando un proyecto
    \item
      Conociendo la estructura del proyecto
    \item
      Servidor de pruebas
    \item
      Configuración del proyecto
    \item
      Sincronizando la base de datos
  \end{itemize}
\end{frame}

\section{Demostración}
\begin{frame}
  \frametitle{Ahora comienza la diversión}
  \begin{itemize}[<+->]
    \item
      Analizando el contenido de la Base de Datos
    \item
      Proyecto vs. Aplicación
    \item
      Creación de la aplicación
    \item
      Definiendo nuestro modelo
    \item
      Activando nuestra aplicación
  \end{itemize}
\end{frame}

\section{Demostración}
\begin{frame}
  \frametitle{Ahora comienza la diversión}
  \begin{itemize}[<+->]
    \item
      Activar la interfaz administrativa
    \item
      Estableciendo las URL
    \item
      Uso de \alert{Generic View}
    \item
      Plantillas
  \end{itemize}
\end{frame}

\begin{frame}
  \frametitle{¡Gracias!}
  \begin{description}
    \item
      [Nombre:] Milton Mazzarri
    \item
      [IRC:] \alert{milmazz} en \alert{irc.freenode.net}, \alert{milmazz}
      en \alert{irc.oftc.net}
    \item 
      [blog:] \url{http://www.milmazz.com}
    \item
      [e-mail:]
      \href{mailto:milmazz@debian.org.ve}{milmazz@debian.org.ve}
  \end{description}
\end{frame}


% preguntas frecuentes
% generic views
% MTV (Model-Template-View)
% demostracion
% configuracion del proyecto
% establecimiento del modelo de datos
% urls.py
% generic views

\end{document}
