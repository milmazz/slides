% Copyright (c) 2010 Milton Mazzarri <milmazz@gmail.com>
% All rights reserved.
%
% Redistribution and use in source and binary forms, with or without
% modification, are permitted provided that the following conditions
% are met:
% 1. Redistributions of source code must retain the above copyright
%    notice, this list of conditions and the following disclaimer.
% 2. Redistributions in binary form must reproduce the above copyright
%    notice, this list of conditions and the following disclaimer in the
%    documentation and/or other materials provided with the distribution.
% 3. The name of the author may not be used to endorse or promote products
%    derived from this software without specific prior written permission.
%
% THIS SOFTWARE IS PROVIDED BY THE AUTHOR ``AS IS'' AND ANY EXPRESS OR
% IMPLIED WARRANTIES, INCLUDING, BUT NOT LIMITED TO, THE IMPLIED WARRANTIES
% OF MERCHANTABILITY AND FITNESS FOR A PARTICULAR PURPOSE ARE DISCLAIMED.
% IN NO EVENT SHALL THE AUTHOR BE LIABLE FOR ANY DIRECT, INDIRECT,
% INCIDENTAL, SPECIAL, EXEMPLARY, OR CONSEQUENTIAL DAMAGES (INCLUDING, BUT
% NOT LIMITED TO, PROCUREMENT OF SUBSTITUTE GOODS OR SERVICES; LOSS OF USE,
% DATA, OR PROFITS; OR BUSINESS INTERRUPTION) HOWEVER CAUSED AND ON ANY
% THEORY OF LIABILITY, WHETHER IN CONTRACT, STRICT LIABILITY, OR TORT
% (INCLUDING NEGLIGENCE OR OTHERWISE) ARISING IN ANY WAY OUT OF THE USE OF
% THIS SOFTWARE, EVEN IF ADVISED OF THE POSSIBILITY OF SUCH DAMAGE.

\documentclass{beamer}

\usetheme{JuanLesPins}
 %AnnArbor
 %Antibes
 %Berlin
 %Boadilla
 %Copenhagen
 %Darmstadt
 %default
 %Dresden
 %Frankfurt
 %Hannover
 %Ilmenau
 %JuanLesPins
 %Madrid
 %Rochester
 %Warsaw
\useoutertheme{}
%default
%infolines
%miniframes
%shadow
%sidebar
%smoothbars
%smooththtree
%split
%tree
\useinnertheme{}
%circles
%default
%inmargin
%rectangles
%rounded
\usefonttheme{}
%default
%professionalfonts
%serif
%estructurebold
%estructureitalicserif
%estructuresmallcapsserif
\usecolortheme{}
%batross
%beaver
%beetle
%crane
%default
%dolphin
%dove
%fly
%lily
%orchid
%rose
%seagull
%seahorse
%sidebartab
%structure
%whale
%wolverine

% Desactivar la barra de navegacion
%\setbeamertemplate{navigation symbols}{}

\usepackage[spanish]{babel}
\usepackage[utf8]{inputenc}
\usepackage{amsmath}

% Redefinicion de teorema, definicion, ejemplo
\newtheorem{teorema}{Teorema}
\newtheorem{definicion}{Definición}
\newtheorem{ejemplo}{Ejemplo}

\hypersetup{colorlinks=true,linkcolor=red}

\title[\texttt{dbconfig-common}]{Mejores prácticas para el \\
empaquetamiento de aplicaciones Debian}
\subtitle{\texttt{dbconfig-common}}
\author[Milton Mazzarri]{Milton Mazzarri \\
\href{mailto:milmazz@gmail.com}{milmazz@gmail.com}
}
\institute[PDVSA]{Petróleos de Venezuela, S.A.}
\date{25 de Agosto de 2008}

\begin{document}

\begin{frame}[plain,label=intro]
\titlepage
\end{frame}

\section{Usando \texttt{dbconfig-common} en sus paquetes}

\frame
{
  \frametitle{Resumen de actividades}

  \begin{itemize}
    \item
      Proveer el \emph{script} o código para establecer la base de datos.
    \item
      Enlazar con los \emph{scripts} del mantenedor de la biblioteca.
    \item
      Ejecutar \texttt{dbconfig-common}.
  \end{itemize}
}

\frame
{
	\frametitle{Beneficios de utilizar \texttt{dbconfig-common}}

    \texttt{dbconfig-common} se encargará de todas las preguntas relacionadas
    con \texttt{debconf}.

  \begin{itemize}
    \item
      Creación de bases de datos.
    \item
      Creación de usuarios de bases de datos.
    \item
      Actualizar, remover, purgar la lógica.
    \item
      En resumen, \texttt{dbconfig-common} se encargará de hacerle fácil la
      vida al administrador local y al mantenedor del paquete.
  \end{itemize}
}

\frame
{
  \frametitle{Enlazando las bibliotecas que harán el trabajo}

  \begin{itemize}
    \item
      Los \emph{scripts} \texttt{config}, \texttt{postinst}, \texttt{prerm},
      \texttt{postrm} del paquete deberá enlazarlos con las bibliotecas que
      harán el trabajo por usted.
    \item
      El paso anterior no necesita hacerlo en el \emph{script}
      \texttt{preinst}.
    \item
      Si su paquete no está usando \texttt{debconf}, lo hará ahora, las
      bibliotecas de \texttt{debconf} deberá enlazarlas primero.
    \item
      Deberá activar en el fichero \texttt{rules} el asistente \emph{debhelper}
      \texttt{dh\_installdebconf}.
  \end{itemize}
}

\begin{frame}[fragile]
  \frametitle{Creando nuestro primer paquete con \texttt{dbconfig-common}}

  \begin{ejemplo}
    \begin{verbatim}
$ ls .
  dbconfig_test-0.1
  dbconfig_test-0.1.orig.tar.gz
$ ls dbconfig_test-0.1/
  pgsql.sql
$ cd dbconfig_test-0.1
$ dh_make -e milmazz@gmail.com -p dbconfig-test
    \end{verbatim}
  \end{ejemplo}
\end{frame}

\section{Directorio \texttt{debian}}
\subsection{Fichero \texttt{changelog}}

\begin{frame}[fragile]
  \frametitle{Editando los ficheros creados por \texttt{debhelper}}

  Para esto nos ubicamos dentro del directorio \texttt{debian}.

  \begin{ejemplo}
    \begin{semiverbatim}
\tiny{$ cat changelog }
  dbconfig-test (0.1-1) unstable; urgency=low

   * Versión inicial.

 -- Milton R. Mazzarri S. <milmazz@gmail.com>  Tue, 19 Aug 2008 08:21:09 -0430
    \end{semiverbatim}
  \end{ejemplo}
\end{frame}

\subsection{Fichero \texttt{control}}

\begin{frame}[fragile]
  \frametitle{Editando los ficheros creados por \texttt{debhelper}}

  \begin{ejemplo}
    \begin{semiverbatim}
\tiny{$ cat control}
Source: dbconfig-test
Section: misc
Priority: optional
Maintainer: Milton R. Mazzarri S. <milmazz@gmail.com>
Build-Depends: debhelper (>= 7)
Standards-Version: 3.7.3
Homepage: http://www.milmazz.com.ve

Package: dbconfig-test
Architecture: any
Depends: ${shlibs:Depends}, ${misc:Depends}, \textbf{dbconfig-common}
Description: Paquete de demostración para el taller
 Demostración de dbconfig-common para el taller.
    \end{semiverbatim}
\end{ejemplo}
\end{frame}

\subsection{Fichero \texttt{install}}

\begin{frame}[fragile]
  \frametitle{Editando los ficheros creados por \texttt{debhelper}}

  \begin{ejemplo}
    \begin{semiverbatim}
\tiny{$ cat install}
pgsql.sql /usr/share/dbconfig-common/data/dbconfig-test/install/pgsql
    \end{semiverbatim}
  \end{ejemplo}
\end{frame}

\subsection{Fichero \texttt{config}}

\begin{frame}[fragile]
  \frametitle{Editando los ficheros creados por \texttt{debhelper}}

  \begin{ejemplo}
    \begin{semiverbatim}
\tiny{$ cat config}
#!/bin/sh

set -e

if [ -f /usr/share/debconf/confmodule ]; then
  . /usr/share/debconf/confmodule
fi

if [ -f /usr/share/dbconfig-common/dpkg/config.pgsql ]; then
  . /usr/share/dbconfig-common/dpkg/config.pgsql
fi

dbc_dbuser="galba"
dbc_dbname="galba"
dbc_go dbconfig-test $@
    \end{semiverbatim}
  \end{ejemplo}
\end{frame}

\subsection{Fichero \texttt{postinst}}

\begin{frame}[fragile]
  \frametitle{Editando los ficheros creados por \texttt{debhelper}}

  \begin{ejemplo}
    \begin{semiverbatim}
\tiny{$ cat postinst }
#!/bin/sh

set -e

if [ -f /usr/share/debconf/confmodule ]; then
  . /usr/share/debconf/confmodule
fi

if [ -f /usr/share/dbconfig-common/dpkg/postinst.pgsql ]; then
  . /usr/share/dbconfig-common/dpkg/postinst.pgsql
fi

dbc_pgsql_createdb_encoding=\"UTF8"
dbc_go dbconfig-test $@
    \end{semiverbatim}
  \end{ejemplo}
\end{frame}

\subsection{Fichero \texttt{postrm}}

\begin{frame}[fragile]

  \frametitle{Editando los ficheros creados por \texttt{debhelper}}

  \begin{ejemplo}
    \begin{semiverbatim}
\tiny{$ cat postrm}
#!/bin/sh

set -e

if [ -f /usr/share/debconf/confmodule ]; then
  . /usr/share/debconf/confmodule
fi

if [ -f /usr/share/dbconfig-common/dpkg/postrm.pgsql ]; then
  . /usr/share/dbconfig-common/dpkg/postrm.pgsql
fi

dbc_go dbconfig-test $@
    \end{semiverbatim}
  \end{ejemplo}
\end{frame}

\subsection{Fichero \texttt{prerm}}

\begin{frame}[fragile]
  \frametitle{Editando los ficheros creados por \texttt{debhelper}}

  \begin{ejemplo}
    \begin{semiverbatim}
\tiny{$ cat prerm}
#!/bin/sh

set -e

if [ -f /usr/share/debconf/confmodule ]; then
  . /usr/share/debconf/confmodule
fi

if [ -f /usr/share/dbconfig-common/dpkg/prerm.pgsql ]; then
  . /usr/share/dbconfig-common/dpkg/prerm.pgsql
fi

dbc_go dbconfig-test $@
    \end{semiverbatim}
  \end{ejemplo}
\end{frame}

\subsection{Fichero \texttt{rules}}

\begin{frame}

  \frametitle{Editando los ficheros creados por \texttt{debhelper}}

  \begin{itemize}
    \item
      Habilitar la variable de ambiente \texttt{DH\_VERBOSE=1}.
    \item
      Activar la opción \texttt{dh\_installdebconf} en la sección \texttt{binary-arch}.
  \end{itemize}
\end{frame}

\end{document}
